Secure Multiparty Computation (MPC) provides parties with the ability to compute aggregates and functions over a set of inputs. The inputs of each party are considered secret and are not revealed during the course of the computation. The only information released by the computation is the final output (could be different to each party). Clearly, MPC can have great impact over many domains where the input data is sensitive but aggregating data from multiple source can provide beneficial information to participants. However, MPC performs much slower than evaluating the functions over the inputs directly in the open (by a trusted party). This added cost comes from the added cost of communication and the additional complexity of performing operations over shares of data instead of the complete input. \newline

In this paper we develop a solution for computing distances in a network. The network consists of many sub-network each belonging to a party, as well as public connections between some number of ``gateway'' nodes in these sub-networks. Each participating party will know only the distance between each of its own nodes and the closest ``threat'' node. Threat could be used to express any useful metric like security leaks, publicly accessible nodes, crashed nodes, etc.. \newline

We provide two techniques to improve the efficiency of the computation, we believe these techniques can be applied to a variety of problems. 
\begin{enumerate}
\item \emph{Reducing the size of the MPC:} We split the computation into stages. The function to be computed (Network distance) is applied locally by each party. Selected pieces of the local outputs (Gateway nodes) are then secretly shared between the parties and consist the input of the MPC stage. The gateway nodes need to be augmented with additional distance information to properly represent their local sub-networks in the secure computation. The results of the MPC stage are put back into the local sub-network, and the parties apply another round of local computation to compute the final distance for each node.
\item \emph{Optimizing the MPC by unrolling the loops:} The network distance problem is iterative in nature, each iteration propagates the ``threat'' metric from each node to its neighbors. Nodes take the minimum threat from each neighbor. This require us to compute a large number of min operations in every iteration, the result of which are feed into the next iteration. Given that min is an expensive operation to perform in MPC. We unroll the loops into symbolic expression, and optimize that expression by exploiting the properties of min and addition. 
\end{enumerate}

The rest of the paper is as follows: In section 1 we provide the detailed problem definition along with the security assumptions and guarantees. In section 2 we discuss some of the difficulties that make this problem hard to solve efficiently in MPC, and briefly discuss our proposed solution to each. Section 3 describes the stages of the computation. Section 4 details our technique to optimize the MPC stage by optimizing its equivalent symbolic expression. We provide details about our implementation and some benchmarks in Section 5. We discuss future work and other possible applications of our technique in section 6. We conclude in section 7.